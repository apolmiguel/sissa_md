\section{L1 - Phonon review and density functional perturbation theory}

The course on Electronic Structure II will focus on the following topics:
\begin{itemize}
    \item Linear response theory: dynamical matrix (phonon frequencies), as well as thermodynamics.
    \item Relativsitic effects (spin-orbit coupling) 
\end{itemize}

\subsection{Review of Phonon equations of motion and the dynamical matrix}

\subsubsection{Definitions for solids and the dynamical equation of motion}
We start with solids. Solids are defined with a regular \hb{position} that is periodic.
\begin{align}
    \*R_I &= \*R_\mu + \*d_s;\qquad S = 1,...,N_\text{at} \label{eq:esii01_position0}
\end{align}
$\*R_I$ are the positions of the lattice sites, with index $I={\mu,s}$, $\mu$ being the lattice site, and $s$ being the atoms in the lattice.  $\*d_s$ are the positions of the atoms in the unit cell. We may consider a typical position vector with the following form,
\begin{align}
    R_{n_1,n_2,n_3} &= n_1\*a_1 + n_2\*a_2 + n_3\*a_3
\end{align}

We consider the \hb{displacements} at which each atom $I$ is displaced as,
\begin{align}
    U_{I\alpha}(t) &= U_{\mu s\alpha}(t) \label{eq:esii01_displacement0}
\end{align}
The atoms $s$ have their own corresponding mass $M_s$ and charge $Z_s$.

Depending on the size of the system we consider, the values $n_i\in\left[0,N_i-1\right], i=1,2,3$ to be infinite if we make $N_i$ large enough. In the lattice of $N$ sites, we may consider $N=N_1\times N_2\times N_3$. The number of degrees of freedom in the phonon problem is (dof) is then $3\times N\times N_\text{at}$. We also consider $\Omega$ to be the volume of 1 unit cell, and $V=N\Omega$ to be the total volume of the system we are considering. 

The position vector as a function of time is,
\begin{align}
    \*R_I(t) &= \*R_I^0 + U_I(t) \label{eq:esii01_position_v_t}
\end{align}

As we focus on linear response theory, we assume that \ho{vibrations/displacements are small}. The assumption makes sense in the low temperature regime. Mathematically, $|U_I(t)|$ is small.

To discuss linear response, we set up a Hamiltonian $H=T+V$. The kinetic energy is,
\begin{align}
    T&= \sum_{I\alpha} \frac{P_{I\alpha}^2(t)}{2M_s} \label{eq:esii01_kinetic0}
\end{align}
The expression for the potential energy is simplified through the \ho{Born-Oppenheimer} approximation\footnote{Electrons move under the effects of $V_\text{ion}$ and vice-versa.}. 
\begin{align}
    W &= E_\text{tot}\left(\lbrace \*R_I^0 + \*U_I(t)\rbrace\right)
    \label{eq:esii01_potentiale0}
\end{align}
Since we consider small vibrations, we can use the Taylor expansion to describe $W$ and keep until the $2^\circ$ term.
\begin{align}
    W &\approx E_\text{tot}\left(\lbrace \*R_I^0\rbrace\right) + \sum_{I\alpha} \frac{\partial E_\text{tot}}{\partial U_{I\alpha}}\evat_{U=0}U_{I\alpha} + \frac{1}{2} \sum_{IJ,\alpha\beta} \frac{\partial^2 E_\text{tot}}{\partial U_{I\alpha}\partial U_{J\beta}}\evat_{U_{I\alpha}U_{J\beta}}U_{I\alpha}U_{J\beta} \notag\\
    W &\approx E_\text{tot}\left(\lbrace \*R_I^0\rbrace\right) + \frac{1}{2} \sum_{IJ,\alpha\beta} \frac{\partial^2 E_\text{tot}}{\partial U_{I\alpha}\partial U_{J\beta}}\evat_{U=0}U_{I\alpha}U_{J\beta} \label{eq:esii01_potentiale1}
\end{align}
The second line argues that the first order term is $0$ since the force is zero at the equilibrium position. We then combine Eqs. \ref{eq:esii01_kinetic0} and \ref{eq:esii01_potentiale1} to show the Hamiltonian under small vibrations,
\begin{align}
    H &= \sum_{I\alpha} \frac{P_{I\alpha}^2(t)}{2M_s} + E_\text{tot}\left(\lbrace \*R_I^0\rbrace\right) + \frac{1}{2} \sum_{IJ,\alpha\beta} \frac{\partial^2 E_\text{tot}}{\partial U_{I\alpha}\partial U_{J\beta}}\evat_{U=0}U_{I\alpha}U_{J\beta}
\end{align}

From classical mechanics, we have the following Hamilton equations of motion,
\begin{align}
    \frac{dU_{I\alpha}}{dt} &= \frac{\partial H}{\partial P_{I\alpha}} = \frac{P_{I\alpha}}{M_s} \label{eq:esii01_hamilton1} \\
    \frac{dP_{I\alpha}}{dt} &= -\frac{\partial H}{\partial U_{I\alpha}} = -\sum_{IJ,\alpha\beta} \frac{\partial^2 E_\text{tot}}{\partial U_{I\alpha}\partial U_{J\beta}}\evat_{U=0}U_{J\beta} \label{eq:esii01_hamilton2} 
\end{align}
Combining both equations lead to the \hb{dynamical equation of motion}\footnote{With $3\times N\times N_\text{at}$ degrees of freedom.}.
\begin{align}
    M_s\frac{d^2U_{I\alpha}}{dt^2} &= -\sum_{J\beta} \frac{\partial^2E_\text{tot}}{\partial U_{I\alpha}\partial U_{J\beta}} U_{J\beta} \label{eq:esii01_dynamicaleom}
\end{align}
The equation of motion in Eq. \ref{eq:esii01_dynamicaleom} may be solved through the use of phonons. 

\subsubsection{Phonons and the dynamical matrix}
We start the solution by having an ansatz about the displacement.
\begin{align}
    U_{\mu s\alpha} &= U_{s\alpha}(\*q)e^{i\*q\cdot\*R_\mu}
\end{align}
\hr{Review}: Certain values of $\*q$ indicate different behaviors due to phases. If $\*q=0$ ($\Gamma$ point), the phase factor is just $1$, and whatever happens in the unit cell repeats. If $\*q=\*G/2$, and $G=(2\pi/a); \*q\cdot\*R = n\pi$, then the phase factor is $e^{in\pi}$ which means that you need twice the unit cell (super cell) to be able to get the periodic behavior properly. Further dividing the $\*q$ vector requires a larger supercell to obtain the proper periodicity. Phase information is important because calculations in metal response to vibrations are typically done in one unit cell only. 

Now we include some $q,t$-dependent normalization terms for the displacement.
\begin{align}
    U_{\mu s\alpha} &= \frac{A(\*q,t)}{\sqrt{M_s}}\tilde{U}_{s\alpha}(\*q)e^{i\*q\cdot\*R_\mu} \label{eq:esii01_dispwithnorm}
\end{align}
Substituting $U_{\mu s\alpha}$ to the equation of motion in Eq. \ref{eq:esii01_dynamicaleom},
\begin{align}
    M_s\frac{d^2A(\*q,t)}{dt^2}\frac{\tilde{U}_{s\alpha}(\*q)}{\sqrt{M_s}} e^{i\*q\cdot\*R_\mu} &= -\sum_{\mu s\beta} \frac{\partial^2E_\text{tot}}{\partial U_{\mu s\alpha} \partial U_{\nu s'\beta}} \frac{A(\*q,t)}{\sqrt{M_s'}}\tilde{U}_{s'\beta}(\*q) e^{i\*q\cdot\*R_\nu} \notag\\
    \frac{d^2A(\*q,t)}{dt^2}\tilde{U}_{s\alpha}(\*q) &= -A(\*q,t)\sum_{\nu s'\beta}\frac{e^{-i\*q\cdot\*R_\mu}}{\sqrt{M_sM_{s'}}}\frac{\partial^2E_\text{tot}}{\partial U_{\mu s\alpha} \partial U_{\nu s'\beta}}e^{i\*q\cdot\*R_\nu}\tilde{U}_{s'\beta}(\*q) \label{eq:esii01_dynamicaleom2}
\end{align}
Note that the phase factors, when combined, will only be dependent on the distances $\*R_\mu - \*R_\nu$. This means that the summation can be done without relying on the index $\mu$, as seen on the RHS of Eq. \ref{eq:esii01_dynamicaleom2}. From this equation, we can also define the \hb{dynamical matrix}. 
\begin{align}
    D_{ss'\alpha\beta}(\*q) &= \frac{1}{\sqrt{M_sM_{s'}}} \sum_\nu e^{-i\*q\cdot\*R_\mu}\frac{\partial^2E_\text{tot}}{\partial U_{\mu s\alpha} \partial U_{\nu s'\beta}}e^{i\*q\cdot\*R_\nu}\tilde{U}_{s'\beta}(\*q) \label{eq:esii01_dynamat0}
\end{align}
The equation of motion in Eq. \ref{eq:esii01_dynamicaleom2} may then be rewritten as,
\begin{align}
    \frac{d^2A(\*q,t)}{dt^2}\tilde{U}_{s\alpha}(\*q) &= -A(\*q,t)\sum_{s'\beta} D_{ss'\alpha\beta}(\*q)\tilde{U}_{s'\beta}(\*q) \label{eq:esii01_dynamicaleom3}
\end{align}

Now we point out some comments about the dyamical matrix. 
\begin{enumerate}[label=\arabic *.]
    \item $D_{ss'\alpha\beta}$ should be \hb{invariant under translations} of $G$ in $k$-space. 
    \begin{align}
        D_{ss'\alpha\beta}\left(\*q+\*G\right) &= \frac{1}{\sqrt{M_sM_{s'}}}\sum_\nu e^{-i(\*q+\*G)\cdot\*R_\mu} \frac{\partial^2E_\text{tot}}{\partial U_{\mu s\alpha}\partial U_{\nu s'\beta}}e^{i(\*q+\*G)\cdot\*R_\nu} \notag\\
        D_{ss'\alpha\beta}\left(\*q+\*G\right) &= \frac{1}{\sqrt{M_sM_{s'}}}\sum_\nu e^{-i\*q \cdot\*R_\mu} \frac{\partial^2E_\text{tot}}{\partial U_{\mu s\alpha}\partial U_{\nu s'\beta}} e^{i\*q \cdot\*R_\nu} e^{i\*G\cdot(R_\nu-R_\mu)} = D_{ss'\alpha\beta}(\*q) \label{eq:esii01_dynamat_tinvar}
    \end{align}
    \item $D_{ss'\alpha\beta}$ is \hb{Hermitian}. \hg{Show explicitly as an exercise.}
    \begin{align}
        D^*_{s's\beta\alpha}(\*q) &= D_{ss'\alpha\beta}(\*q) \label{eq:esii01_dynamat_hermitian}
    \end{align}
    \item $D_{ss'\alpha\beta}$ is \hb{even}.
    \begin{align}
        D_{ss'\alpha\beta}(-\*q) &= D^*_{s's\beta\alpha}(\*q)
    \end{align}
\end{enumerate}

With the properties of $D_{ss'\alpha\beta}$, we can effectively \hr{reduce the degrees of freedom} to $3N_\text{at}$. With Eq. \ref{eq:esii01_dynamicaleom3}, we can form the eigenvalue equation for the dynamical matrix.
\begin{align}
    \sum_{s'\beta} D_{ss'\alpha\beta}(\*q)\*e_{s'\beta}^\eta(\*q) &= \omega_{q\eta}^2(\*q)\*e_{s\alpha}^\eta(\*q)
    \label{eq:esii01_dynamat_eigvaleq}
\end{align}
The index $\eta$ labels the different eigenvectors/eigenvalues to the solution. To solve Eq. \ref{eq:esii01_dynamat_eigvaleq}, we must look for solutions of $A(\*q,t)$ satisfying the simple harmonic oscillator differential equation.
\begin{align}
    \frac{d^2A(\*q,t)}{dt^2} &= -\omega_{q\eta}A^\eta(\*q,t) \label{eq:esii01_A_sho}
\end{align}
The solution to the problem is:
\begin{align}
    A(\*q,t) &= A^{q\eta}\sin\left(\omega_{q\eta}+\delta^{q\eta}\right) \label{eq:esii01_A_sho_sol}
\end{align}
The general idea then is to use DFT to write the dynamical matrix $D_{ss'\alpha\beta}$, then solve the eigenvalue problem in Eq. \ref{eq:esii01_dynamat_eigvaleq}, and finally get the frequencies of the phonon modes. \hr{Not sure where this came from}: From the eigenvectors satisfying,
\begin{align}
    \sum_{s\alpha} \*e^\eta_{s\alpha}(\*q) \*e^{\eta'}_{s'\beta}(\*{q'
    }) &= \delta^{mm'}\delta(\*q-\*{q'})
\end{align}
We may consider the eigenbasis of the dynamical matrix to be complete.
\begin{align}
    \sum_{\eta} \*e^\eta_{s\alpha}(\*q) \*e^{eta'}_{s'\beta}(\*{q'
    }) &= \delta^{ss'}_{\alpha\beta}
\end{align}

To compute the (2nd) derivatives of the dynamical matrix, we start to look at perturbative methods.
\begin{align*}
    D_{ss'\alpha\beta}(\*q) &= \frac{1}{\sqrt{M_sM_{s'}}} \sum_\nu e^{-i\*q\cdot\*R_\mu}\frac{\partial^2E_\text{tot}}{\partial U_{\mu s\alpha} \partial U_{\nu s'\beta}}e^{i\*q\cdot\*R_\nu}\tilde{U}_{s'\beta}(\*q) 
\end{align*}
We then call $\lambda=U_{\mu s\alpha}$ and $\mu=U_{\nu s'\beta}$\footnote{Forgive the abuse of notation for now.} as the perturbation parameters of the Hamiltonian. 

\subsection{Density functional perturbation theory}
General idea outlined below. \hg{Rewrite explicitly in here with enough time, but handwriting notes is fine for now.}
\begin{itemize}
    \item Compute for $E_\text{tot}$ leading to the density $n(R)$.
    \item Use the variational principle to calculate for $\frac{\partial^2E_\text{tot}}{\partial\lambda \partial\nu}$ and use the Gellman-Feynman theorem to find an expression for the derivative terms. 
    \item Extract $\frac{dn}{d\mu}$  and introduce braket notation to express the derivative differently.
    \item Go back to minimization, and recover the Kohn-Sham (KS) equations. Expand $V_{KS}(\mu),\varepsilon_i(\mu)$, and $\psi_i(\mu,r)$ as series expansions around $\mu=0$ up to linear order.
    \item Extract the linear orders from the expansion and calculate for the other derivatives. 
    \item Projection of $\frac{dn}{d\mu}$ in the unoccupied/conduction band as the ones in the occupied band will just be equal to zero. 
    \item Next time, show that $\frac{\partial^2V_\text{loc}(\*r)}{\partial U_{s\alpha}\partial U_{s'\beta}}$ as lattice periodic. 
\end{itemize}